\documentclass[12pt]{letter}
\usepackage{amsmath, amssymb}
\usepackage[margin=1.6in]{geometry}
\usepackage{fancyvrb}
\usepackage{color}
\usepackage{tabularx}
\usepackage[scaled=.90]{helvet}
\usepackage[parfill]{parskip}

\newcounter{section}
\newcounter{subsection}[section]
\setcounter{secnumdepth}{3}
\makeatletter
\newcommand\section{\@startsection{section}{1}{\z@}%
                                   {-2.5ex \@plus -1ex \@minus -.2ex}%
                                   {2.3ex \@plus.2ex}%
                                   {\normalfont\large\textsf}}
\newcommand\subsection{\@startsection{subsection}{2}{\z@}%
                                     {-2.25ex\@plus -1ex \@minus -.2ex}%
                                     {1.5ex \@plus .2ex}%
                                     {\normalfont\textsf}}
\renewcommand\thesection{\@arabic\c@section}
\renewcommand\thesubsection{\thesection.\@arabic\c@subsection}
\makeatother

% LMAO "address"
\address{%
  \textsf{User Testing Report}\\
  {To: Ana Leighe Cooke}\\
  {From: Team Git}
}

\begin{document}
\begin{letter}{Professor Cooke,}%

\opening{}

The purpose of this memo is to cover the objectives, users, tasks, and results of our user testing.

\section*{Objectives}

Our intended goal is for our manual to be primarily used as an introduction the basics of version control with Git for beginners.  Our secondary goal is for our manual to serve as a quick reference for Git commands for intermediate to advanced users.

Our user tests were split between two groups: beginner and intermediate users of Git.  Our objective with the beginner group was to test the immediate use of our manual for someone with little to no knowledge of Git.  Our objective for the intermediate group was to test how useful our manual would serve as a quick reference.

\section*{Users and Testing Procedure}

\subsection*{Users}

\begin{center}
\begin{tabular}{llll}
User & Level & Method & Contributor\\
\hline
Tim Becker & intermediate & coaching & Evan\\
Cole Heathershaw & intermediate & think aloud & Fred\\
Riley Xu & intermediate & think aloud & Sunny\\
Xinni Wu & beginner & coaching & Ziyang\\
Di Wong & intermediate & think aloud & Ziyang\\
George Situ & beginner & coaching & Nishad\\
\end{tabular}
\end{center}

\subsection*{Testing Procedure}

For beginner-level users, we opted for the coaching method.  Beginners were handed a computer with a directory with several files and a list of tasks to accomplish.  The manual was available as their primary source of help.

For intermediate-level users, we used the think-aloud protocol which is better for getting feedback from more experienced users since they can read the instructions and simultaneously compare it to their previous experience using and learning Git.

We used the whole draft on our users, since it is intended for random-access reading with users being able to pick out sections they need.

\section*{Testing Results and Feedback}

\subsection*{Tim Becker}

\subsection*{Cole Heathershaw}

Overall, Cole was off-put by the overall tone of the manual.  In particular, he found some sections were ``too-personal."  Part of the inconsistency across sections in terms of tone is due to the way we allocated sections to individual group members.  Furthermore, Cole disagreed with the inclusion of the ``Why use Git?" section, which he considered unnecessary considering that most people opening a manual know why they are using the manual.

A larger portion of his feedback stemmed from his differentiation between guides and manuals, with manuals addressing more expert users and guides being closer to an introduction.  In this regard, our manual was more of a guide since our target audience consists of beginner and intermediate users of Git.

For some other feedback points, Cole mentioned the following:
\begin{itemize}
    \item Get rid of the use of ``we"
    \item Add warnings to some sections such as ``Deleting branches"
    \item Fix dangling sections
    \item Be more direct instead of suggesting in some sections (i.e. removing ``perhaps," ``want,")
    \item In ``Merging branches," what is the word ``This" referring to?
    \item The Problem/Solution/Description separation works sometimes, but in some sections the problem statement is so short to the point that it is redundant with the section header.
\end{itemize}
\subsection*{Riley Xu}

We used the think-aloud protocol with Riley. Since he is an experienced Git user, we had him read through the manual and voice his opinions on the various sections. His feedback was more focused on the content of the manual, but he also gave information on the organization of the manual. He was reasonably impressed with the manual and liked the visuals.

Riley mentioned the following:
\begin{itemize}
  \item Explain the difference between rebasing and merging clearly.
  \item Emphasize that while switching branches the changes made to current branch are
  reverted.
  \item Specify when to delete branches and when to merge them.
  \item Fix grammatical errors.
  \item Include a cheatsheet at the end i.e. a list of common git commands and
  references to them in the manual.
  \item More detail needed in git add (e.g. git add ., git add -A).
  \item Did not introduce Git commit hashes, therefore the section on reverting 
  changes and git history can be confusing to readers.
  \item In the images in installation section, zoom in on the relevant section.
  \item It was unclear how to actually make a Git alias.
  
\end{itemize}

\subsection*{Xinni Wu}

\subsection*{Di Wong}

\subsection*{George Situ}

\section*{Revisions}

\end{letter}
\end{document}