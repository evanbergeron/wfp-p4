\documentclass{amsart}
\usepackage{amsmath, amssymb}
\usepackage{mathpartir}
\usepackage{listings}
\usepackage{fancyvrb}
\usepackage{color}
\usepackage{bussproofs}
\usepackage{pgf, tikz}
\usetikzlibrary{arrows, automata}

\fvset{%
  fontsize=\small,
  numbers=left
}

\makeatletter
\renewcommand\subsection{\@startsection{subsection}{2}%
  \z@{-.5\linespacing\@plus-.7\linespacing}{.5\linespacing}%
  {\normalfont\scshape}}
\renewcommand\subsubsection{\@startsection{subsubsection}{3}%
  \z@{.5\linespacing\@plus.7\linespacing}{-.5em}%
  {\normalfont\scshape}}
\makeatother

\title{Git User Testing Plan and Task Schedule}
\author{Evan Bergeron\\
Frederick Lee\\
Ziyang Wang\\
Sunny Gakhar\\
Nishad Gothoskar}


\begin{document}
\maketitle

\section*{User Testing Plan}

We plan on targeting individuals that are either new to or experienced with Git.  The reason for this is that we envision our manual primary usage to be an introduction to the basics of Git for beginners as well as a quick reference for Git commands for intermediate to advanced users.

The reason we chose the coaching method for beginners is to better simulate a first-time experience with using Git.  Beginners will be handed a computer with a directory with several files and a list of tasks to accomplish.  The manual will be available as their primary source of help.

The think-aloud protocol is better for getting feedback from more experienced users since they can read the instructions and simultaneously compare it to their previous experience using and learning Git.

\subsection*{Intended user testers}

\begin{center}
\begin{tabular}{llll}
User & Level & Method & Contributor\\
\hline
Tim Becker & beginner & coaching & Evan\\
Cole Heathershaw & intermediate & think aloud & Fred\\
Riley Xu & intermediate & think aloud & Sunny\\
Xinni Wu & beginner & coaching & Ziyang\\
Di Wong & intermediate & think aloud & Ziyang\\
George Situ & beginner & coaching & Nishad\\
\end{tabular}
\end{center}



We plan on testing the whole draft on our users, since it is attended to be modularized for easier skimming with users being able to pick out sections that they need at the time.

\newpage
\section*{Task Schedule}

The booklet will be written in \LaTeX with collaboration being handled with GitHub.  We have the following task allocation and schedule:

\subsection*{Task List}
\begin{center}
\begin{tabular}{llll}
Task & Est. Time & Contributor(s) & Due Date\\
\hline
What is Git? & 2 hours & Sunny & April 19\\
Installing Git & 2 hours & Sunny & April 19\\
Initializing repository & 2 hours & Sunny & April 19\\
Adding files to version control & 2 hours & Ziyang & April 19\\
Committing changes & 2 hours & Ziyang & April 19\\
Reverting changes & 2 hours & Ziyang & April 19\\
Branching & 2 hours & Nishad & April 19\\
Merging branches & 2 hours & Nishad & April 19\\
Ignoring files & 2 hours & Fred & April 19\\
Rebasing & 2 hours & Fred & April 19\\
Cherrypicking & 2 hours & Fred & April 19\\
Collaboration with GitHub & 2 hours & Nishad & April 19\\
General workflows & 2 hours & Evan & April 19\\
Vim workflows & 3 hours & Evan & April 19\\
User Testing & 2 hours & Everyone & April 28\\
Peer Review & 3 hours & Everyone & April 28\\
\end{tabular}
\end{center}



\end{document}