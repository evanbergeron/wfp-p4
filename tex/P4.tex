\documentclass{amsart}
\usepackage{amsmath, amssymb}
%\usepackage{mathpartir}
\usepackage{listings}
\usepackage{fancyvrb}
\usepackage{color}
\usepackage{bussproofs}
\usepackage{pgf, tikz}
\usetikzlibrary{arrows, automata}

\fvset{%
  fontsize=\small,
  numbers=left
}

\makeatletter
\renewcommand\subsection{\@startsection{subsection}{2}%
  \z@{-.5\linespacing\@plus-.7\linespacing}{.5\linespacing}%
  {\normalfont\scshape}}
\renewcommand\subsubsection{\@startsection{subsubsection}{3}%
  \z@{.5\linespacing\@plus.7\linespacing}{-.5em}%
  {\normalfont\scshape}}
\makeatother

\setcounter{tocdepth}{3}
\makeatletter
\def\l@subsection{\@tocline{2}{0pt}{2.5pc}{5pc}{}}
%Make Chapter disapear in ToC
\renewcommand\tocchapter[3]{%
  \indentlabel{\@ifnotempty{#2}{\ignorespaces#2.\quad}}#3%
}
\newcommand\@dotsep{4.5}
\def\@tocline#1#2#3#4#5#6#7{\relax
  \ifnum #1>\c@tocdepth % then omit
  \else
    \par \addpenalty\@secpenalty\addvspace{#2}%
    \begingroup \hyphenpenalty\@M
    \@ifempty{#4}{%
      \@tempdima\csname r@tocindent\number#1\endcsname\relax
    }{%
      \@tempdima#4\relax
    }%
    \parindent\z@ \leftskip#3\relax \advance\leftskip\@tempdima\relax
    \rightskip\@pnumwidth plus1em \parfillskip-\@pnumwidth
    #5\leavevmode\hskip-\@tempdima{#6}\nobreak
    \leaders\hbox{$\m@th\mkern \@dotsep mu\hbox{.}\mkern \@dotsep mu$}\hfill
    \nobreak
    \hbox to\@pnumwidth{\@tocpagenum{#7}}\par
    \nobreak
    \endgroup
  \fi}

\title{Git Instruction Manual}
\author{Evan Bergeron\\
Frederick Lee\\
Ziyang Wang\\
Sunny Gakhar\\
Nishad Gothoskar}


\begin{document}
\maketitle

\tableofcontents
\section*{Introduction}

This is a manual on how to set up Git on your computer and set up a basic Git workflow. The document covers installing and setting up Git and how to work with Git. The majority of this document is primarily intended for users who want to learn Git to use it in a fast-paced setting such as a hackathon, this document is also useful as a reference for experienced Git users who want to refer to some specifc concept or command which they need.

\section*{What is Git}

\section*{Initializing Repository}

\section*{Adding files to version control}

\section*{Committing changes}

\section*{Reverting changes}

\section*{Branching}

\section*{Merging Branches}

\section*{Ignoring Files}

\section*{Rebasing}

\section*{Cherrypicking}

\section*{Collaboration with Github}

\section*{General workflows}

\subsection*{Easily Reordering Commits}

We’ve all been there. You pull from master, take several minutes to
clean up the various merge conflicts, and then are ready to push. You
pull one last time before pushing, and what do you know - someone has
pushed again in the last couple minutes.

Previously, my workflow for this situation would look something like
\begin{verbatim}
$ git checkout -b tmp
$ git checkout master
$ git reset --hard HEAD~
$ git cherry-pick tmp
\end{verbatim}

This works fine, but there’s a much easier way -- one that involves
very little typing. We can simply say
\[
  \texttt{git rebase -i HEAD~2}
\]
This will open an interactive git rebasing session (the \texttt{-i}
stands for interactive). The window will display something along the
lines of
\begin{verbatim}
pick 370e221 Commit one
pick c342396 Commit two
\end{verbatim}
In whichever text editor we’re in, we may simply reorder these lines
to reorder the commits. Much shorter!

\subsection*{Adding partial files}
I just used this, actually. Suppose you’ve changed a single file
\texttt{foo.c} in different sections, and each of these changes are
logically different. For instance, maybe you refactor some function
\texttt{foo}, while at the same time fixing a bug in function
\texttt{bar}. Rather than create a separate branch and manaully edit
the files, we can simply say
\[
  \texttt{git add -p foo.c}
\]
This will bring up an interactive prompt. It will automatically cycle
through all the different areas of the diff, asking you if you want to
stage each section. You may hit \texttt{y} or \texttt{n} for yes or
no.

Once you’re done adding the subset of changes you want to commit, you
can double-check you have the right changes staged by saying
\[
  \texttt{git diff --cached}
\]
Once you’re sure that you’re good to go, just commit your changes as
normal. You can repeat this process for the remaining changes. (Or
just do a normal \texttt{git add} at this point).

\subsection*{Git aliases}
Git aliases are a good way to save yourself a lot of typing. I
frequently want to see the git commit history, but don’t especially
care about the body of each commit. Here’s the command I’ve added to
my configuration:
\[
  \texttt{git config --global alias.l "log --oneline"}
\]
I can then just type “\texttt{git l}” to see a one line log of this
commit history. If there are several long commands you use frequently,
this can be a great way to save yourself some time.

\section*{Vim workflows}

At the time of writing, perhaps the most feature complete vim-git
plugin is Tim Pope’s “vim-fugitive.” Consequently, we will assume
usage of this plugin throughout the entire vim workflows tutorial.

\subsection*{Installing vim-fugitive}

There are a number of ways to install vim-fugitive. The one suggested
by Tim Pope is as follows:
\begin{verbatim}
$ cd ~/.vim/bundle
$ git clone git://github.com/tpope/vim-fugitive.git
$ vim -u NONE -c "helptags vim-fugitive/doc" -c q
\end{verbatim}
Vundle is a great plugin manager for vim -- if you use this, you may
simply add the line
\[
  \texttt{Plugin 'tpope/vim-fugitive}
\]
to your vimrc and run the \texttt{PluginInstall} command.

\section*{Emacs workflows}

Similarly to vim-fugitive for vim, “Magit” is (at the time of
writing), the most feature-complete git wrapper for emacs. We will
thus assume usage of this package.

\subsection*{Installing Magit}

\end{document}