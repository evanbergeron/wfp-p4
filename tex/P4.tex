\documentclass{amsart}
\usepackage{amsmath, amssymb}
%\usepackage{mathpartir}
\usepackage{listings}
\usepackage{fancyvrb}
\usepackage{color}
\usepackage{bussproofs}
\usepackage{pgf, tikz}
\usetikzlibrary{arrows, automata}

\fvset{%
  fontsize=\small,
  numbers=left
}

\makeatletter
\renewcommand\subsection{\@startsection{subsection}{2}%
  \z@{-.5\linespacing\@plus-.7\linespacing}{.5\linespacing}%
  {\normalfont\scshape}}
\renewcommand\subsubsection{\@startsection{subsubsection}{3}%
  \z@{.5\linespacing\@plus.7\linespacing}{-.5em}%
  {\normalfont\scshape}}
\makeatother

\title{Git Instruction Manual}
\author{Evan Bergeron\\
Frederick Lee\\
Ziyang Wang\\
Sunny Gakhar\\
Nishad Gothoskar}


\begin{document}
\maketitle

\section*{Introduction}

This is a manual on how to set up Git on your computer and set up a basic Git workflow. The document covers installing and setting up Git and how to work with Git. The majority of this document is primarily intended for users who want to learn Git to use it in a fast-paced setting such as a hackathon, this document is also useful as a reference for experienced Git users who want to refer to some specifc concept or command which they need.

\newpage

\section*{Contents}

% When you are done with a section, remove the TODO and put the page number instead

\subsection*{Table of Contents}
\begin{center}
	\begin{tabular}{ll}
		Topic & Page Number\\
		\hline
		What is Git? & TODO\\
		Installing Git & TODO \\
		Initializing repository & TODO\\
		Adding files to version control & TODO\\
		Committing changes & TODO\\
		Reverting changes & TODO\\
		Branching & TODO\\
		Merging branches & TODO\\
		Ignoring files & TODO\\
		Rebasing & TODO\\
		Cherrypicking & TODO\\
		Collaboration with GitHub & TODO\\
		General workflows & TODO\\
		Vim workflows & TODO\\

	\end{tabular}
\end{center}

\newpage

\section*{What is Git}

\newpage

\section*{Initializing Repository}

\newpage

\section*{Adding files to version control}

\newpage

\section*{Committing changes}

\newpage

\section*{Reverting changes}

\newpage

\section*{Branching}

\newpage

\section*{Merging Branches}

\newpage

\section*{Ignoring Files}

\newpage

\section*{Rebasing}

\newpage

\section*{Cherrypicking}

\newpage

\section*{Collaboration with Github}

\newpage

\section*{General workflows}

\subsection*{Easily Reordering Commits}

We’ve all been there. You pull from master, take several minutes to clean up the various merge conflicts, and then are ready to push. You pull one last time before pushing, and what do you know - someone has pushed again in the last couple minutes.

Previously, my workflow for this situation would look something like
\begin{verbatim}
$ git checkout -b tmp
$ git checkout master
$ git reset --hard HEAD~
$ git cherry-pick tmp
\end{verbatim}

This works fine, but there’s a much easier way -- one that involves
very little typing. We can simply say
\[
  \texttt{git rebase -i HEAD~2}
\]
This will open an interactive git rebasing session (the \texttt{-i}
stands for interactive). The window will display something along the
lines of
\begin{verbatim}
pick 370e221 Commit one
pick c342396 Commit two
\end{verbatim}
In whichever text editor we’re in, we may simply reorder these lines
to reorder the commits. Much shorter!

\section*{Vim workflows}

At the time of writing, perhaps the most feature complete vim-git
plugin is Tim Pope’s “vim-fugitive.” Consequently, we will assume
usage of this plugin throughout the entire vim workflows tutorial.

\subsection*{Installing vim-fugitive}

There are a number of ways to install vim-fugitive. The one suggested
by Tim Pope is as follows:
\begin{verbatim}
$ cd ~/.vim/bundle
$ git clone git://github.com/tpope/vim-fugitive.git
$ vim -u NONE -c "helptags vim-fugitive/doc" -c q
\end{verbatim}
Vundle is a great plugin manager for vim -- if you use this, you may
simply add the line
\[
  \texttt{Plugin 'tpope/vim-fugitive}
\]
to your vimrc and run the \texttt{PluginInstall} command.

\section*{Emacs workflows}

Similarly to vim-fugitive for vim, “Magit” is (at the time of
writing), the most feature-complete git wrapper for emacs. We will
thus assume usage of this package.

\subsection*{Installing Magit}

\end{document}