\documentclass{amsart}
\usepackage{amsmath, amssymb}
%\usepackage{mathpartir}
\usepackage{listings}
\usepackage{fancyvrb}
\usepackage{color}
\usepackage{bussproofs}
\usepackage{pgf, tikz}
\usetikzlibrary{arrows, automata}

\fvset{%
  fontsize=\small,
  numbers=left
}

\makeatletter
\renewcommand\subsection{\@startsection{subsection}{2}%
  \z@{-.5\linespacing\@plus-.7\linespacing}{.5\linespacing}%
  {\normalfont\scshape}}
\renewcommand\subsubsection{\@startsection{subsubsection}{3}%
  \z@{.5\linespacing\@plus.7\linespacing}{-.5em}%
  {\normalfont\scshape}}
\makeatother

\title{Git Instruction Manual}
\author{Evan Bergeron\\
Frederick Lee\\
Ziyang Wang\\
Sunny Gakhar\\
Nishad Gothoskar}


\begin{document}
\maketitle

\section*{Introduction}

This is a manual on how to set up Git on your computer and set up a basic Git workflow. The document covers installing and setting up Git and how to work with Git. The majority of this document is primarily intended for users who want to learn Git to use it in a fast-paced setting such as a hackathon, this document is also useful as a reference for experienced Git users who want to refer to some specifc concept or command which they need.

\newpage

\section*{Contents}

% When you are done with a section, remove the TODO and put the page number instead

\subsection*{Table of Contents}
\begin{center}
	\begin{tabular}{ll}
		Topic & Page Number\\
		\hline
		What is Git? & TODO\\
		Installing Git & TODO \\
		Initializing repository & TODO\\
		Adding files to version control & TODO\\
		Committing changes & TODO\\
		Reverting changes & TODO\\
		Branching & TODO\\
		Merging branches & TODO\\
		Ignoring files & TODO\\
		Rebasing & TODO\\
		Cherrypicking & TODO\\
		Collaboration with GitHub & TODO\\
		General workflows & TODO\\
		Vim workflows & TODO\\

	\end{tabular}
\end{center}

\newpage

\section*{What is Git}

\newpage

\section*{Initializing Repository}

\newpage

\section*{Adding files to version control}

\newpage

\section*{Committing changes}

\newpage

\section*{Reverting changes}

\newpage

\section*{Branching}

\newpage

\section*{Merging Branches}

\newpage

\section*{Ignoring Files}

\newpage

\section*{Rebasing}

\newpage

\section*{Cherrypicking}

\newpage

\section*{Collaboration with Github}

\newpage

\section*{General workflows}

\newpage

\section*{Vim workflows}

\end{document}